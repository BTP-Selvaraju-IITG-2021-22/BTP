\chapter{Introduction}

In this paper, our main goal is to analyze the application of game theory to networks of queues in order to optimize a payoff function associated with the queueing game. We explore non-cooperative queueing games in Chapter 2 by presenting a model and stating theorems that prove/disprove the existence of Nash Equilibrium for certain sub-classes of queueing games.
In Chapter 3, we have introduced an algorithmic approach to check the existence of pure strategy (mixed also?) Nash Equilibria for non-cooperative N-player games on a generalized network of queues, with a predefined strategy space for each player. We have also explored and analyzed the Best-response algorithm which shows that, for a game with a continuous strategy space, a pure-strategy Nash Equilibrium always exists.
In Chapter 4, we conclude the paper with a brief overview of our findings; and provide an array of avenues which would look to explore and work on, in the future.



\section{Queueing Theory}

Some text here ...

\begin{definition}\label{abc1}
$M/M/1$ Queue
	An M/M/1 queue is a single-server queue, and according to the Kendall's notation, has arrival rate($\lambda$) following the Markovian($M$) distribution, which means the inter-arrival times of customers entering the queue are exponential. The service rate($\mu$) of the queue	is also Markovian($M$) and hence is an exponential service time. The maximum number of customers in the queue at the same time is unbounded or infinite.
\end{definition}
\vspace{2mm}
\begin{theorem}
The expected waiting time for an M/M/1 queue with arrival rate $\lambda$ and service rate $\mu$ is equal to $\frac{1}{\mu - \lambda}$.
\end{theorem}

\begin{proof}
Let n be the number of customers at a given time, in the queue. We can make the flow-balance equations for n $\ge$ 1 and for the state with no customers as follows:
\begin{center}
$(\lambda + \mu)p_n = \mu p_{n+1} + \lambda p_{n-1}$ \\
$\lambda p_0 = \mu p_1$ \\
\end{center}
where $p_i$ is the long-term fraction of time with $i$ customers in the system. The following figure shows a state diagram for the number of customers in the system at a time along with the rates of transition to the next or previous state.
<insert figure of rate transition diagram here>

\begin{center}
$L = \frac{\rho}{1-\rho} = \frac{\lambda}{\mu - \lambda}$
\end{center}
Using Little's Law ($L = \lambda W$), we get,
\begin{center}
	 $W = \frac{L}{\lambda} = \frac{1}{\mu - \lambda}$
\end{center}
\end{proof}
\vspace{20mm}

\begin{corollary}
A corollary to the theorem is....
\end{corollary}

\begin{remark}
Some remark.......
\end{remark}


You may have to type many equations inside the text.  The equation can be typed as below.
\begin{equation}\label{eqn1}
f(x) = \frac{x^2-5x+2}{e^x - 2} = {y^5-3 \over e^x-2} %\nonumber
\end{equation}

This can be referred as (\ref{eqn1}) and so on.....

You may have to type a set of equations.  For this you may proceed as given below.
\begin{eqnarray}
f(x) &=& e^{1+2(x-a)} + \ldots   \nonumber   \\
  &=& \log(x+a) + \sin(x+y) + \cdots  \label{eqn2}
\end{eqnarray}

% Note: \nonumber will suppress the eqn number in the above.
% You can type comments like this starting with % as here.

You may have to cite the articles.  You may do so as \cite{laan} and so on.....
Note that you have already created the `bib.bib' file and included the entry with the above name. Only
then you can cite it as above.

\section{Game Theory}
\begin{definition}\label{abc2}
Some definition....
\end{definition}

\begin{remark}
Some remark.......
\end{remark}

\subsection{Subsection name}

\begin{theorem}
Some theorem.......
\end{theorem}

\begin{proof}
Proof is as follows.... By Definition \ref{abc1}
\end{proof}


\begin{figure}[h]

[The figure will be displayed here.]

\caption{The correlation coefficient as a function of $\rho$}
\end{figure}


