\documentclass[12pt,a4paper]{report}

\usepackage{amsthm,amssymb,mathrsfs,setspace}%amsmath, latexsym,footmisc

% \usepackage{pstcol}
% \usepackage{play}
\usepackage{epsfig}
%\usepackage[grey,times]{quotchap}
\usepackage[nottoc]{tocbibind}
\renewcommand{\chaptermark}[1]{\markboth{#1}{}}
\renewcommand{\sectionmark}[1]{\markright{\thesection\ #1}}
%

\input xy
\xyoption{all}


\theoremstyle{plain}
\newtheorem{theorem}{Theorem}[section]
\newtheorem{lemma}[theorem]{Lemma}
\newtheorem{corollary}[theorem]{Corollary}
\newtheorem{proposition}[theorem]{Proposition}

\theoremstyle{definition}
\newtheorem{definition}[theorem]{Definition}
\newtheorem{example}[theorem]{Example}
\newtheorem{notation}[theorem]{Notation}

\theoremstyle{remark}
\newtheorem{remark}[theorem]{Remark}

\renewcommand{\baselinestretch}{1.5}




\begin{document}

%\pagenumbering{arabic} \setcounter{page}{1}

% --------------- Title page -----------------------

\begin{titlepage}
\enlargethispage{3cm}

\begin{center}

\vspace*{-2cm}

\textbf{\Large ANALYSIS OF QUEUEING SYSTEMS USING GAME THEORY}

\vfill

 A Project Report Submitted \\
for the Course \\[1cm]

{\bf\Large\ MA498 Project ~I }\\[.1in]

 \vfill

{\large \emph{by}}\\[5pt]
{\large\bf {Karan Gupta}}\\[5pt]
{\large (Roll No. 180123064)}\\
{\large\bf {Ashish Kumar Barnawal}}\\[5pt]
{\large (Roll No. 180123006)}

\vfill
\includegraphics[height=2.5cm]{iitglogo}

\vspace*{0.5cm}

{\em\large to the}\\[10pt]
{\bf\large DEPARTMENT OF MATHEMATICS} \\[5pt]
{\bf\large \mbox{INDIAN INSTITUTE OF TECHNOLOGY GUWAHATI}}\\[5pt]
{\bf\large GUWAHATI - 781039, INDIA}\\[10pt]
{\it\large November 2021}
\end{center}

\end{titlepage}

\clearpage

% --------------- Certificate page -----------------------
\pagenumbering{roman} \setcounter{page}{2}
\begin{center}
{\Large{\bf{CERTIFICATE}}}
\end{center}
%\thispagestyle{empty}


\noindent
This is to certify that the work contained in this project report entitled 
“Analysis of Queueing Systems using Game Theory” submitted by Karan Gupta (Roll No. 180123064 and Ashish Kumar Barnawal (Roll No.: 180123006) 
to the Department of Mathematics, Indian Institute of Technology Guwahati towards partial requirement of
Bachelor of Technology in Mathematics and Computing has been carried out by him/her under
my supervision. \\

\noindent
It is also certified that this report is a survey work based on the references
in the bibliography.\\

OR\\

\noindent
It is also certified that, along with literature survey,
a few new results are established/computational implementations have been carried
out/simulation studies have been carried out/empirical analysis
has been done by the student under the project.\\

\noindent
Turnitin Similarity: \_\_ \%
%

\vspace{4cm}

\noindent Guwahati - 781 039 \hfill (Prof. N. Selvaraju)

\noindent November 2021 \hfill Project Supervisor

\clearpage

% --------------- Abstract page -----------------------
\begin{center}
{\Large{\bf{ABSTRACT}}}
\end{center}


The main aim of the project ........

\clearpage



\tableofcontents
\clearpage
\listoffigures
\listoftables


\newpage

\pagenumbering{arabic}
\setcounter{page}{1}

% =========== Main chapters starts here. Type in separate files and include the filename here. ==
% ============================


%%%%%%%%%%%%%%%%%%%
%%%% Chapter 1 %%%%
%%%%%%%%%%%%%%%%%%%
\chapter{Introduction}

In this paper, our main goal is to analyze the application of game theory to networks of queues in order to optimize a payoff function associated with the queueing game. We explore non-cooperative queueing games in Chapter 2 by presenting a model and stating theorems that prove/disprove the existence of Nash Equilibrium for certain sub-classes of queueing games.
In Chapter 3, we have introduced an algorithmic approach to check the existence of pure strategy (mixed also?) Nash Equilibria for non-cooperative N-player games on a generalized network of queues, with a predefined strategy space for each player. We have also explored and analyzed the Best-response algorithm which shows that, for a game with a continuous strategy space, a pure-strategy Nash Equilibrium always exists.
In Chapter 4, we conclude the paper with a brief overview of our findings; and provide an array of avenues which would look to explore and work on, in the future.



\section{Queueing Theory}

Some text here ...

\begin{definition}\label{abc1}
$M/M/1$ Queue
	An M/M/1 queue is a single-server queue, and according to the Kendall's notation, has arrival rate($\lambda$) following the Markovian($M$) distribution, which means the inter-arrival times of customers entering the queue are exponential. The service rate($\mu$) of the queue	is also Markovian($M$) and hence is an exponential service time. The maximum number of customers in the queue at the same time is unbounded or infinite.
\end{definition}
\vspace{2mm}
\begin{theorem}
The expected waiting time for an M/M/1 queue with arrival rate $\lambda$ and service rate $\mu$ is equal to $\frac{1}{\mu - \lambda}$.
\end{theorem}

\begin{proof}
Let n be the number of customers at a given time, in the queue. We can make the flow-balance equations for n $\ge$ 1 and for the state with no customers as follows:
\begin{center}
$(\lambda + \mu)p_n = \mu p_{n+1} + \lambda p_{n-1}$ \\
$\lambda p_0 = \mu p_1$ \\
\end{center}
where $p_i$ is the long-term fraction of time with $i$ customers in the system. The following figure shows a state diagram for the number of customers in the system at a time along with the rates of transition to the next or previous state.
<insert figure of rate transition diagram here>

\begin{center}
$L = \frac{\rho}{1-\rho} = \frac{\lambda}{\mu - \lambda}$
\end{center}
Using Little's Law ($L = \lambda W$), we get,
\begin{center}
	 $W = \frac{L}{\lambda} = \frac{1}{\mu - \lambda}$
\end{center}
\end{proof}
\vspace{20mm}

\begin{corollary}
A corollary to the theorem is....
\end{corollary}

\begin{remark}
Some remark.......
\end{remark}


You may have to type many equations inside the text.  The equation can be typed as below.
\begin{equation}\label{eqn1}
f(x) = \frac{x^2-5x+2}{e^x - 2} = {y^5-3 \over e^x-2} %\nonumber
\end{equation}

This can be referred as (\ref{eqn1}) and so on.....

You may have to type a set of equations.  For this you may proceed as given below.
\begin{eqnarray}
f(x) &=& e^{1+2(x-a)} + \ldots   \nonumber   \\
  &=& \log(x+a) + \sin(x+y) + \cdots  \label{eqn2}
\end{eqnarray}

% Note: \nonumber will suppress the eqn number in the above.
% You can type comments like this starting with % as here.

You may have to cite the articles.  You may do so as \cite{laan} and so on.....
Note that you have already created the `bib.bib' file and included the entry with the above name. Only
then you can cite it as above.

\section{Game Theory}
\begin{definition}\label{abc2}
Some definition....
\end{definition}

\begin{remark}
Some remark.......
\end{remark}

\subsection{Subsection name}

\begin{theorem}
Some theorem.......
\end{theorem}

\begin{proof}
Proof is as follows.... By Definition \ref{abc1}
\end{proof}


\begin{figure}[h]

[The figure will be displayed here.]

\caption{The correlation coefficient as a function of $\rho$}
\end{figure}





%%%%%%%%%%%%%%%%%%%
%%%% Chapter 2 %%%%
%%%%%%%%%%%%%%%%%%%
\chapter{Introduction}

In this paper, our main goal is to analyze the application of game theory to networks of queues in order to optimize a payoff function associated with the queueing game. We explore non-cooperative queueing games in Chapter 2 by presenting a model and stating theorems that prove/disprove the existence of Nash Equilibrium for certain sub-classes of queueing games.
In Chapter 3, we have introduced an algorithmic approach to check the existence of pure strategy (mixed also?) Nash Equilibria for non-cooperative N-player games on a generalized network of queues, with a predefined strategy space for each player. We have also explored and analyzed the Best-response algorithm which shows that, for a game with a continuous strategy space, a pure-strategy Nash Equilibrium always exists.
In Chapter 4, we conclude the paper with a brief overview of our findings; and provide an array of avenues which would look to explore and work on, in the future.



\section{Queueing Theory}

Some text here ...

\begin{definition}\label{abc1}
$M/M/1$ Queue
	An M/M/1 queue is a single-server queue, and according to the Kendall's notation, has arrival rate($\lambda$) following the Markovian($M$) distribution, which means the inter-arrival times of customers entering the queue are exponential. The service rate($\mu$) of the queue	is also Markovian($M$) and hence is an exponential service time. The maximum number of customers in the queue at the same time is unbounded or infinite.
\end{definition}
\vspace{2mm}
\begin{theorem}
The expected waiting time for an M/M/1 queue with arrival rate $\lambda$ and service rate $\mu$ is equal to $\frac{1}{\mu - \lambda}$.
\end{theorem}

\begin{proof}
Let n be the number of customers at a given time, in the queue. We can make the flow-balance equations for n $\ge$ 1 and for the state with no customers as follows:
\begin{center}
$(\lambda + \mu)p_n = \mu p_{n+1} + \lambda p_{n-1}$ \\
$\lambda p_0 = \mu p_1$ \\
\end{center}
where $p_i$ is the long-term fraction of time with $i$ customers in the system. The following figure shows a state diagram for the number of customers in the system at a time along with the rates of transition to the next or previous state.
<insert figure of rate transition diagram here>

\begin{center}
$L = \frac{\rho}{1-\rho} = \frac{\lambda}{\mu - \lambda}$
\end{center}
Using Little's Law ($L = \lambda W$), we get,
\begin{center}
	 $W = \frac{L}{\lambda} = \frac{1}{\mu - \lambda}$
\end{center}
\end{proof}
\vspace{20mm}

\begin{corollary}
A corollary to the theorem is....
\end{corollary}

\begin{remark}
Some remark.......
\end{remark}


You may have to type many equations inside the text.  The equation can be typed as below.
\begin{equation}\label{eqn1}
f(x) = \frac{x^2-5x+2}{e^x - 2} = {y^5-3 \over e^x-2} %\nonumber
\end{equation}

This can be referred as (\ref{eqn1}) and so on.....

You may have to type a set of equations.  For this you may proceed as given below.
\begin{eqnarray}
f(x) &=& e^{1+2(x-a)} + \ldots   \nonumber   \\
  &=& \log(x+a) + \sin(x+y) + \cdots  \label{eqn2}
\end{eqnarray}

% Note: \nonumber will suppress the eqn number in the above.
% You can type comments like this starting with % as here.

You may have to cite the articles.  You may do so as \cite{laan} and so on.....
Note that you have already created the `bib.bib' file and included the entry with the above name. Only
then you can cite it as above.

\section{Game Theory}
\begin{definition}\label{abc2}
Some definition....
\end{definition}

\begin{remark}
Some remark.......
\end{remark}

\subsection{Subsection name}

\begin{theorem}
Some theorem.......
\end{theorem}

\begin{proof}
Proof is as follows.... By Definition \ref{abc1}
\end{proof}


\begin{figure}[h]

[The figure will be displayed here.]

\caption{The correlation coefficient as a function of $\rho$}
\end{figure}


%%%%%%%%%%%%%%%%%%%
%%%% Chapter 3 %%%%
%%%%%%%%%%%%%%%%%%%
\chapter{Algorithms for Nash Equilibria of Queueing games}

Introductory lines...



\section{Section-1 Name}
\begin{definition}\label{abc5}
Some definition....
\end{definition}

\begin{remark}
Some remark.......
\end{remark}



\begin{theorem}
Some theorem.......
\end{theorem}

\begin{proof}
Proof is as follows....
\end{proof}

\section{Section-2 Name}
\begin{definition}\label{abc6}
Some definition....
\end{definition}

\begin{remark}
Some remark.......
\end{remark}

\subsection{Subsection name}

\begin{theorem}
Some theorem.......
\end{theorem}

\begin{proof}
Proof is as follows....
\end{proof}



%%%%%%%%%%%%%%%%%%%
%%%% Chapter 4 %%%%
%%%%%%%%%%%%%%%%%%%
\chapter{Future Work}

Introductory lines...



\section{Section-1 Name}
\begin{definition}\label{abc7}
Some definition....
\end{definition}

\begin{remark}
Some remark.......
\end{remark}



\begin{theorem}
Some theorem.......
\end{theorem}

\begin{proof}
Proof is as follows....
\end{proof}

\section{Section-2 Name}
\begin{definition}\label{abc8}
Some definition....
\end{definition}

\begin{remark}
Some remark.......
\end{remark}

\subsection{Subsection name}

\begin{theorem}
Some theorem.......
\end{theorem}

\begin{proof}
Proof is as follows....
\end{proof}


\nocite{laan}\nocite{rqhassin}\nocite{stidham}\nocite{gross}\nocite{osborne}\nocite{ferreira}

\bibliographystyle{plain}
\bibliography{bib.bib}

\end{document}

