\chapter{Conclusion and Future Extensions}

\section{Conclusion}
In this paper, we have tried to emphasize on the application of game theory to optimize the routing strategy across a network of queues when, multiple rational Decision-makers are involved. Extensive research has been done in this field, especially in recent years; a good account of which can be found in \cite{rqhassin}, Chapter 8.

We have explored congestion games and the Rosenthal theorem (1973) which proves the finiteness of improvement steps in a congestion game. We have also studied the proofs for the existence of a pure-strategy Nash Equilibrium in an N-player queueing game with equal arrival rates.

We have introduced and illustrated with examples, an algorithm that gives the payoff matrix for general queueing games from which Nash Equilibrium can be computed. The existence of Nash Equilibrium need not guarantee the minimization of mean sojourn time for each player. Further topics of research are discussed in the next section.


\section{Future Extension}
After studying the existence of pure-strategy Nash-Equilibria for one of the subclasses in a discrete strategy space in \cite{laan}, we would like to explore further for more subclasses and prove or disprove the existence of pure/mixed strategy Nash-Equilibrium for them. The subclassese include but are not limited to
\begin{itemize}
    \item N-player games with equal arrival rate on multiserver queues ($M/M/c$ queues)
    \item 3-player games with identical service rates
    \item 2-player games with different distributions for the arrival and service rates of queues (like $M/E_k/1$, $M/D/1$) and addrd waiting cost in queues.
    \item Games with payoff functions for players other than their mean sojourn time (MST) like maximizing the MST for all other player or minimize waiting time at the final queue.
    \item $N$-player games with certain shapes of queueing networks without any constraint on arrival rate or service rate.
\end{itemize}


% \begin{remark}
% Some remark.......
% \end{remark}



% \begin{theorem}
% Some theorem.......
% \end{theorem}

% \begin{proof}
% Proof is as follows....
% \end{proof}


